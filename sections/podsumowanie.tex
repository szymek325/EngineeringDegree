Wynikiem niniejszej pracy inżynierskiej jest skonstruowany obiekt oraz układ regulacji temperatury. Pomimo pewnych trudności napotkanych na początku budowy układu regulacji, po prawidłowym doborze radiatorów do ogniwa Peltiera, prace nad projektem nabrały szybszego tempa. Główne prace zostały zakończone na tyle szybko, że podjęto decyzję o dodaniu do pracy aplikacji mobilnej, kosztem rezygnacji z mniej interesujących elementów. Aplikacja została dostosowana do urządzeń mobilnych z oprogramowaniem Android. W celu przyspieszenia regulacji temperatury, w ostatecznej wersji obiekt został pomniejszony. Niniejsza praca pozwoliła na rozwinięcie się w obszarach zainteresowań autora oraz na poznanie nowych metod tworzenia aplikacji. Zaprojektowane oprogramowanie sterujące, dobrze radzi sobie z regulacją temperatury w wybranym zakresie. Aplikacja mobilna pozwala na przetestowanie regulatora histerezowego oraz PID, w dowolnej konfiguracji.

W przypadku przyszłych modyfikacji, warto byłoby, aby układ regulacji temperatury został oparty o płytkę STM lub AVR. Ponadto, można by pomyśleć o rozbudowaniu obiektu do postaci makiety domu jednorodzinnego lub biurowca, w którym zostałyby zainstalowane dodatkowe czujniki oraz sterowano by oświetleniem. Ciekawym rozwiązaniem byłoby wykorzystanie protokołu komunikacji ZigBee, z którego zrezygnowano podczas wykonywania projektu.
