Celem pracy było zaprojektowanie i stworzenie konstrukcji obiektu sterowania oraz układu regulacji temperatury. Po wcześniejszej konsultacji z Promotorem, podjęto decyzję o zrezygnowaniu z wykonania układu zasilania oraz o wykorzystaniu części gotowych podzespołów. Aby wzbogacić zakres zadań pracy, za wspólnym porozumieniem podjęto decyzję o utworzeniu aplikacji mobilnej pozwalającej na zdalne sterowanie układem regulacji temperatury.

Układ regulacji został oparty o układ mikroprocesorowy Arduino oraz ogniwo Peltiera. Zakres pracy obejmuje następujące zagadnienia:
\begin{itemize}
\item projekt i konstrukcja obiektu, w którym będzie regulowana temperatura,
\item przedstawienie połączenia elektrycznego układu sterowania temperaturą,
\item prawidłowe oprogramowanie sterowanie ogniwem i pozostałymi podzespołami,
\item utworzenie aplikacji mobilnej pozwalającej na kontrolę temperatury oraz odczyt danych pomiarowych uzyskanych przez układ,
\item nawiązanie komunikacji między układem regulacji, a aplikacją,
\item uruchomienie układu i przeprowadzenie testów działania wybranych regulatorów.
\end{itemize}